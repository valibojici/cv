\documentclass{cv_template}
\begin{document}

\cvHeader
{Valentin-Gabriel Bojici}
{\faEnvelopeO \, valibojici@gmail.com,
\faPhone \, 0728119910,
\faLinkedin \, \href{https://www.linkedin.com/in/valentin-bojici/}{linkedin.com/in/valentin-bojici/},
\faGithub \, \href{https://www.github.com/valibojici}{github.com/valibojici}
}

\section{Educație}

\datedsubsection[c][X][c]
{Licență în informatică}
{Universitatea din București}
{2020 -- 2023}

\workitems
[Licențiat în informatică la Facultatea de Matematică și Informatică --- notă finală de 9,5 din 10.]
[Cursuri relevante: 
    $\bullet$ \textit{Programare Orientata Obiect} (C++)
    $\bullet$ \textit{Strcturi de date si algoritmi}
    $\bullet$ \textit{Dezvoltarea aplicatiilor web} (PHP)
    $\bullet$ \textit{Baze de date} (SQL, PL/SQL)
    $\bullet$ \textit{Grafica pe calculator} (OpenGL, C++)
    $\bullet$ \textit{Machine Learning} (supervised learning)
]

\datedsubsection[c][X][c]
{}
{Zero-to-Hero - Deep Learning for Computer Vision Summer School}
{11-15 Iulie 2023}

\workitems
[Finalizat o școală de vară 5 zile (DigitalStack --- Google) și dobândit cunoștințe în tehnici de computer vision.]
[Obținut locul 2 din 11 echipe într-o competiție Kaggle de computer vision.]



\section{Experiență de lucru}

\datedsubsection[c][X][c]
{Backend PHP Developer, Intern}
{Tremend Software Consulting}
{Iulie 2022 - Septembrie 2022}

\workitems
[Am reușit să fac o impresie favorabilă în timpul stagiului, rezultând o ofertă de muncă pe care am ales să o refuz în favoarea continuării studiilor.]
[Am obținut informații valoroase despre dinamica echipei și metodologiile Agile prin participarea activă la întâlnirile cu clienții și echipa.]
[Sarcini finalizate cu succes la nivel de intern: investigarea și rezolvarea erorilor software, optimizarea performanței prin comprimarea imaginilor și îmbunătățirea răspunsurilor API.]
[Am utilizat framework-ul Magento și am câștigat experiență cu alte framework-uri precum Laravel și Symphony prin proiecte personale.]


\section{Proiecte}

% LICENTA
\datedsubsection[c][X][c]
{Illumination Models Viewer}
{\href{https://www.github.com/valibojici/illumination-models}{github.com/valibojici/illumination-models}}
{2023}

\workitems
[Proiect de licență (Grafică pe calculator) --- o aplicație desktop interactivă scrisă în C++ folosind OpenGL care prezintă diferite modele de iluminare în timp real.]
[Abilități utilizate: tehnici de post-procesare (corecție gamma și detectarea marginilor), programare orientată pe obiecte (C++), algebră liniară (pentru calcule de iluminare \& umbre), integrarea bibliotecilor 3rd party, cum ar fi ImGui, OpenGL, GLFW.]

% HILLSIDE HOTEL
\datedsubsection[c][X][c]
{Hillside Hotel}
{\href{https://www.github.com/valibojici/hillside-hotel}{github.com/valibojici/hillside-hotel}}
{2023}

\workitems
[Proiect full-stack realizat cu NodeJS, Express, GraphQL, MySQL și React pentru un hotel fictiv care permite utilizatorilor să facă rezervări.]
[Abilități utilizate: containerizarea aplicației cu Docker, integrarea plăților cu Stripe, gestionarea stării într-un single-page-app, implementarea și utilizarea unui API GraphQL.]

% POETIC SWiPE
\datedsubsection[c][X][c]
{PoeticSwipe}
{\href{https://github.com/valibojici/PoeticSwipe}{github.com/valibojici/PoeticSwipe}}
{2023}

\workitems
[Aplicație Android realizată folosind Flutter pentru descoperirea poeziilor scurte ale unor autori cunoscuți.]
[Abilități utilizate: menținerea stării la nivel de aplicație (\textit{Provider} package), depedency injection (\textit{GetIt} package), implementarea de mocks pentru resurse (\textit{Mockito} package).]


% ARDUINO
\datedsubsection
[c][X][c]
{Arduino Snake Game}
{\href{https://github.com/valibojici/Snake-Arduino}{github.com/valibojici/Snake-Arduino}}
{2022}

\workitems
[Implementarea jocului clasic de snake pe un Arduino UNO cu afișaj LCD și o matrice LED.]
[Abilități utilizate: prioritizarea gestionării eficiente a memoriei având în vedere memoria foarte redusă, utilizarea eficientă a ecranului LCD pentru afișarea meniului, integrarea hardware.]

\section{Limbaje și tehnologii}
\workitems
[C++, C, Python, Dart, Javascript, SQL, HTML/CSS, PHP]
[Flutter, OpenGL, NodeJS, React, Express, GraphQL, Docker, Bootstrap, PyTorch, NumPy]


\thispagestyle{empty}
\end{document}